\documentclass[a4paper,11pt]{article}
\usepackage{a4wide}
\usepackage[utf8]{inputenc}
\usepackage[T1]{fontenc}
\usepackage[czech]{babel}

% vim: spl=cs:


%postup při řešení, způsob řešení
%dosažené cíle
%zdůvodnění případných změn v řešení projektu (technické změny, nikoliv finanční)
%konkrétní výstupy, další využitelnost
%přínosy projektu, vlastní hodnocení
%tisková zpráva – 2 řádky textu (cca 300 znaků) s odkazem na web řešitele

\title{Analýza dat z hmotnostní spektrometrie za použití strojového učení}
\author{Aleš Křenek (Jiří Novotný)}
\date{\today}

\begin{document}
\maketitle

\section{Anotace}
Cílem projektu je přenesení prototypů software pro zpracování dat z hmotností
spektrometrie metodami strojového učení na výpočetní zdroje e-INFRA CZ a
vyhodnocení chování těchto algoritmů na dostupných architekturách akcelerátorů
GPU. Práce se zaměří na řešení problémů spojených s velkými objemy dat, jako
třeba distribuované výpočty nebo současné využití více akcelerátorů, a které
bez použití zdrojů e-INFRA nebylo možné řešit. Projekt navazuje na širší
činnost společné výzkumné skupiny centra Recetox a Ústavu výpočetní techniky MU,
která se zabývá mimo jiné i vývojem těchto metod. Přínosem pro aplikační oblast
bude posun ve vývoji prototypů k aplikovatelnosti na rozsáhlé a pro
kompetitivní výzkum nezbytné datové sady. Přínosem pro e-INFRA CZ bude zejména
zkušenost s přenesením a provozem této třídy aplikací na infrastruktuře a
případná zpětná vazba potřebná k jejímu dalšímu rozvoji, jako například
vhodnost konkrétního HW či změny SW prostředí.


\section{Postup a způsob řešení}

% co je MS

% identifikované problémy v MS vhodné pro strojové učení

% chybějící peaky ve spektru -- opsané z diplomky

% de-novo identifikace -- co má Adam


% benchmark




\section{Dosažené cíle}

% vyzkoušelo se několik metod doplnění peaků, fungují docela dobře
% Michal na tom udělal bakalářku, cena děkana

% transformery pro de-novo identifikaci, prototyp funguje, Adam bude mít diplomku

\section{Zdůvodnění změn}

% JN hodil ručník

% proto prodloužení

% na školení jsme se vykašlali


\section{Konkrétní výstupy a další využitelnost}

% Vlastní softwarový produkt (prototyp)
%• Benchmark + testovací sada
%• Dokumentace k software včetně shrnutí získaných zkušeností a doporučení pro HW/SW
%prostředí
%• Článek shrnující přínos specifického použití metod strojového učení pro aplikační oblast,
%připravený k odeslání do časopisu nebo na odbornou konferenci.
%• Certifikáty ze školení


% implementace chybějících peaků, experimentálně nasazena do Galaxy UMSA

% kód použitelný jako benchmark (github raims)

% prezentace na Metasemináři, Sitole, e-INFRA

\section{Přínosy projektu a hodnocení}

% po útěku JN jsme to stejně zvládli

\section{Tisková zpráva}

\end{document}
